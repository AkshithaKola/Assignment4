\let\negmedspace\undefined
\let\negthickspace\undefined
\documentclass[journal,12pt,onecolumn]{IEEEtran}
\usepackage{gensymb}
\usepackage{amssymb}
\usepackage[cmex10]{amsmath}
\usepackage{amsthm}
\usepackage[justification=centering]{caption}
\usepackage{bm}
\usepackage{longtable}
\usepackage{enumitem}
\usepackage{mathtools}
 \usepackage{tikz}
\usepackage[breaklinks=true]{hyperref}
\usepackage{listings}
\usepackage{color}                                            %%
\usepackage{array}                                            %%
\usepackage{longtable}                                        %%
\usepackage{calc}          
\usepackage{multirow}                                         %%
\usepackage{hhline}                                           %%
\usepackage{ifthen}                                           %%
\usepackage{lscape}     
\usepackage{multicol}
\usepackage[utf8]{inputenc}
\newcommand{\solution}{\noindent \textbf{Solution: }}
\begin{document}
\vspace{3cm}
\title{Assignment IV (CBSE Class 9 Probability)}
\author{Kola Akshitha}
\maketitle
\textbf{Example 7:} The percentage of marks obtained by a student in the monthly unit tests are given below: \\
\begin{table}[ht!]
\begin{center}
    \normalsize \begin{tabular}{c |c |c |c |c |c }

\textbf{Unit test} & $I$ & $II$ & $III$ & $IV$ & $V$ \\
\hline
\textbf{Percentage of marks obtained} & $69$ & $71$ & $73$ & $68$ & $74$

\end{tabular}
	 
	\vspace{5pt}
\caption{}
\label{table:table1}
\end{center}
	
\end{table}
Based on this data find the probability that the student gets more than 70\% marks in a unit test. \\
\solution\\
Total number of unit tests conducted is 5.\\
The number of unit tests in which the student scored more than 70\% is 3 \\
$\therefore$ 
   Pr(scoring greater than 70\% marks)$ = \frac{3}{5} = 0.6$
\end{document}